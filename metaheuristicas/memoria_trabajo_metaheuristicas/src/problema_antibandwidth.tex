\section{Problema Antibandwidth}

\subsection{Revisión bibliográfica usando SCOPUS}

Para comenzar vamos a realizar una breve revisión bibliográfica del problema utilizando la herramienta SCOPUS.

De cara a buscar los trabajos más relevantes para este problema he utilizado el buscador de SCOPUS, ordenando los resultados por número de citas.

Echando un vistazo rápido podemos ver que de cara a resolver este problema se han realizado pruebas con GRASP, algoritmos meméticos, algoritmos de colonias de abejas, búsqueda tabú, híbridos entre estos algoritmos, entre otras técnicas.


El trabajo más relevante en el área de la computación es GRASP with Path Relinking Heuristics for the
Antibandwidth Problem \cite{antibandwidthGRASP}. En este trabajo se propone una forma de resolver el problema usando un modelo de programación entera, así como utilizando GRASP y re-enlazado de caminos. Estas pruebas dieron muy buenos resultados en los distintos grafos utilizados en la experimentación.

Otro trabajo bastante citado es A hybrid metaheuristic for the cyclic antibandwidth problem \cite{hybridMetaheuristicAntibandwith}. Esta propuesta se basa en un algoritmo híbrido entre un algoritmo de colonias de abejas y una búsqueda tabú. Además de explicar todos los detalles del problema, componentes principales del algoritmo que proponen, así como un pseudocódigo de la implementación, realizan diversos experimentos para comprobar el funcionamiento de dicha propuesta, demostrando que es viable y consigue resolver el problema de una forma bastante eficiente, tanto a nivel de calidad de la solución como de como se recorre el espacio de búsqueda del problema.

Además de estas propuestas, otra que destaca es A memetic algorithm for the cyclic antibandwidth maximization problem \cite{memeticoAntibandwith}. Aquí se propone un algoritmo memético, un híbrido entre un algoritmo genético y una búsqueda local, de cara a resolver el problema. Esta propuesta comenta las mejoras que puede conllevar mezclar la exploración de los algoritmos genéticos con la alta explotación de la búsqueda local para encontrar buenas soluciones. Con respecto a los resultados obtenidos, en este estudio se realizan diversas pruebas con grafos en los que es conocida la solución optima, donde la propuesta consigue unos resultados muy buenos.

Como vemos, para este problema todas las propuestas de mayor interés en la literatura se han centrado en resolver el problema utilizando metaheurísticas, principalmente debido a la complejidad del problema.

\subsection{Representación de una solución}

De cara a representar una solución se utilizará un vector con un tamaño igual al número de nodos del grafo, donde cada posición representará un nodo, y el valor en dicha posición la etiqueta asignada a dicho nodo.

Como nos comentaba el enunciado, estas soluciones deberán maximizar la función $AB(f, G)$, siendo $f$ un etiquetado del grafo $G$.

\subsection{Propuesta greedy}

De cara a resolver el problema utilizando un enfoque greedy, lo que haremos será comenzar con una solución con todos los nodos etiquetados a cero, y con esto, iterar sobre cada nodo, asignando la etiqueta que maximice la función objetivo $AB(f,G)$.

\begin{lstlisting}
Greedy AB (G):
	solucion <- vector de longitud |G|, inicializado a cero

	Para cada elemento x de G:
		solucion[x] <- etiqueta que maximice AB(solucion, G), y etiqueta no está en solucion

	Devolver solucion
\end{lstlisting}

\subsection{Propuesta semi-greedy}

De cara a la propuesta semi-greedy, utilizaremos un enfoque similar a la propuesta greedy, pero en lugar de escoger la etiqueta que maximice la función objetivo para cierto nodo, haremos una ruleta con todas las etiquetas, asignando una mayor probabilidad a las etiquetas que más maximicen la función objetivo para dicho nodo.

\begin{lstlisting}
Semi-Greedy AB (G):
	solucion <- vector de longitud |G|, inicializado a cero

	Para cada elemento x de G:
		candidatos <- lista de etiquetas ordenadas segun maximice AB(solucion, G) y que no esté en solucion

		etiqueta <- escoger un elemento de candidatos usando una ruleta (mayor probabilidad a los candidatos que mas maximicen AB)

		solucion[x] <- etiqueta

	Devolver solucion
\end{lstlisting}

\subsection{Propuesta iterated greedy}

Para la propuesta iterated greedy, lo que haremos será partir de la solución greedy generada, como hemos visto en clase, destruir parcialmente dicha solución, y reconstruirla, de forma que si es mejor que la solución que teníamos nos la quedamos.

De cara a destruir parcialmente una solución lo que haremos será usar un porcentaje para indicar en que cantidad queremos destruir dicha solución, etiquetando a cero de forma aleatoria dicho porcentaje de nodos. Para reconstruir la solución simplemente tomaremos el mismo enfoque que el enfoque greedy, pero esta vez solo con los nodos que están etiquetados a cero (hemos destruido su solución).

\begin{lstlisting}
DestruirSolucion(solucion, porcentajeDestruccion):

	Para cada elemento x de solucion:
		Si aleatorio < porcentajeDestruccion:
			solucion[x] <- 0

	Devolver solucion


ConstruirSolucion(destruida, G):
	Para cada elemento x de solucion, tal que solucion[x] = 0:
		solucion[x] <- etiqueta que maximice AB(solucion, G) que no esté en solucion

	Devolver solucion

Iterated-Greedy AB (G, iteraciones, porcentajeDestruccion):
	solucion <- Greedy AB(G)

	Repetir iteraciones veces:
		destruida <- DestruirSolucion(solucion, porcentajeDestruccion)
		reconstruida <- ConstruirSolucion(destruida, G)

		Si AB(reconstruida, G) > AB(solucion, G):
			solucion <- reconstruida

	Devolver solucion
\end{lstlisting}

Como vemos, con estos algoritmos tenemos una forma rápida de obtener una solución al problema, aunque las soluciones no sean de una calidad muy buena. Aun así, en todos los casos las propuestas tienen un orden de complejidad $O(n)$, a excepción de iterated greedy, que sería $O(n^2)$ en el caso de que el porcentaje de destrucción sea del 100\%, pero en ese caso en realidad estaríamos ante un algoritmo greedy con reinicialización, en el que lanzamos muchas veces el algoritmo greedy y nos quedamos con la mejor solución, nunca mantenemos información entre una solución y otra.

\subsection{Propuesta búsqueda local}

De cara a proponer un algoritmo de búsqueda local primero tenemos que proponer el operador de vecindario que vamos a utilizar.

\subsubsection{Operador de vecino}

Para este problema el operador de vecino es muy simple, tan solo intercambiamos las etiquetas asignadas entre dos nodos seleccionados de forma aleatoria, teniendo en cuenta también que antes no se haya generado dicho cambio, para evitar que al generar vecinos de forma aleatoria se generen varias veces los mismos vecinos.

\begin{lstlisting}[language=python]
Operador_vecino(solucion, parejas_generadas):
	vecino <- solucion
	i <- numero entre 0 y la longitud de solucion
	j <- numero entre 0 y la longitud de solucion

	Mientras la pareja i,j esté en parejas_generadas Y i = j:
		i <- numero entre 0 y la longitud de solucion
		j <- numero entre 0 y la longitud de solucion

	vecino[i] <- solucion[j]
	vecino[j] <- solucion[i]

	parejas_generadas <- insertar la pareja i,j y la pareja j, i

	Devolver vecino y parejas_generadas
\end{lstlisting}

\subsubsection{Búsqueda local}

Para el algoritmo de búsqueda local utilizamos el esquema básico de búsqueda local, en el que partimos de una solución ya generada (en este caso he utilizado el greedy básico, pero podríamos utilizar cualquiera de las otras variantes del greedy), y mientras encontremos vecinos mejores, seguimos iterando sobre el algoritmo. En el momento en el que de una solución no encontremos ningún vecino con mejor función objetivo quiere decir que hemos llegado a un máximo local, y por lo tanto hemos finalizado con la búsqueda.

\begin{lstlisting}[language=python]
Busqueda local(G):
	solucion <- Greedy AB(G)
	hay_mejora <- 1

	Mientras hay_mejora = 1:
		parejas_generadas <- conjunto vacio
		hay_mejora <- 0

		Mientras queden parejas sin generar en parejas_generadas y hay_mejora = 0:
			vecino, parejas_generadas <- Operador_vecino(solucion, parejas_generadas)
			Si AB(vecino, G) > AB(solucion, G):
				hay_mejora <- 1
				solucion <- vecino
	Devolver solucion
\end{lstlisting}

\subsection{Propuesta algoritmo genético}

\subsubsection{Representación de los cromosomas}

\subsubsection{Operador de cruce}

\subsubsection{Operador de mutación}

\subsubsection{Discusión sobre la inicialización de la población}
