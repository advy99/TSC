\section{Descripción del problema de la diversidad máxima}

Este problema se basa en, dado un conjunto inicial $N$ con $n$ elementos, escoger un subconjunto $M$ con $m$ de estos elementos tal que $n > m$ de forma que se maximice la diversidad entre los elementos escogidos.

Este problema se puede formular como maximizar la siguiente función:

\[
MD(X) = \displaystyle\sum_{i=1}^{n - 1} \displaystyle\sum_{j=1}^{n} d_{ij} x_i x_j
\]

Sujeto a:

\[
\displaystyle\sum_{i=1}^{n} x_i = m
\]
\[
x_i = \{0,1\}, i = 1, \dots, n
\]

Donde:

\begin{itemize}
	\item $x = (x_1, \dots, x_n)$ es la representación de una solución al problema, que consiste en un vector binario donde $x_i = 1$ indica que el elemento $i$ ha sido seleccionado.
	\item $d_{ij}$ es la distancia existente entre los elementos $i$ y $j$. Estas distancias son parámetros del problema.
\end{itemize}

De cara a resolver este problema se van a proponer dos métodos, una heurística ad-hoc, específica para resolver este problema, y una búsqueda local.

\newpage

\section{Elementos necesarios para resolver el problema}

De cara a poder resolver el problema primero tenemos que plantear distintos elementos sobre como vamos a tratar el problema para poder buscar las soluciones de este y escoger una de ellas. Aunque la mayoría de estos elementos nos los proporciona el enunciado, vamos a formalizarlos.

\subsection{Representación del problema}

La representación viene dada en el enunciado. Una solución a nuestro problema será un vector binario de tamaño $n$, donde $x_i = 1$ significa que hemos seleccionado el elemento $i$ del conjunto $N$ para que forme parte de $M$.

\[
x = (x_1, \dots, x_n) / x_i = \{0, 1\}, i = 1, \dots, n
\]

\subsection{Restricciones del problema}

En este caso el problema solo tiene una restricción:


\[
\displaystyle\sum_{i=1}^{n} x_i = m
\]
\[
x_i = \{0,1\}, i = 1, \dots, n
\]

Es decir, que no podemos seleccionar más de $m$ elementos.

\subsection{Función objetivo}

La función objetivo se trata de la función $MD(X)$, comentada en la descripción del problema al inicio de este documento.

\subsection{Criterio de comparación entre soluciones}

En este caso el criterio de comparación entre dos soluciones será el máximo del valor de su función objetivo, ya que estamos intentando maximizar el valor de la diversidad.

Una vez definidos estos elementos comunes, vamos a pasar a las formas de resolver el problema.

\newpage

\section{Heuristica ad-hoc}

Para resolver el problema usando una heurística no es necesario definir más elementos concretos, con los elementos comunes es suficiente.

La heurística que he pensado se basa en simplemente ir buscando en la matriz de distancias la pareja $i, j$ con mayor distancia. Si nos quedan más de dos elementos por seleccionar, me es indiferente si uno de ellos se ha seleccionado antes o no siempre que sea la mayor distancia de la matriz, mientras que si queda solo un nodo por seleccionar solo escojo una pareja $i, j$ tal que el elemento $j$ ya esté seleccionado, para asegurarme que se escogen exactamente $m$ elementos. Tras actualizar la solución $x$, hago que la distancia $d_{ij} = 0$ para que no vuelva a escoger esa pareja, y vuelvo a repetir el proceso hasta escoger $m$ elementos.

\begin{lstlisting}[language=python]
Heuristica Ad-hoc(n, m, d):
	x <- Secuencia de n ceros
	numero_seleccionados <- 0

	Mientras numero_seleccionados < m:
		Si m - numero_seleccionados > 2:
			i, j <- buscar en d la posición i, j con mayor valor
			Si x[i] = 0 y x[j] = 0:
				x[i] <- 1
				x[j] <- 1
				numero_seleccionados <- numero_seleccionados + 2
			Si x[i] = 1:
				x[j] <- 1
				numero_seleccionados <- numero_seleccionados + 1
			Si x[j] = 1:
				x[i] <- 1
				numero_seleccionados <- numero_seleccionados + 1
			d[i, j] <- 0
		Si no:
			i, j <- buscar en d la posición i, j con mayor valor, sujeto a que x[j] = 1
			x[i] <- 1
			d[i, j] <- 0
	Devolver x como la solución
\end{lstlisting}

\newpage

\section{Búsqueda local}

En el caso de resolver el problema utilizando una búsqueda local si es necesario definir un nuevo elemento, el operador de vecindario. Este operador será el que, a partir de una solución $x$ al problema, genere una nueva solución $x'$ dentro del vecindario de la solución $x$, es decir, una solución $x'$ parecida a $x$ pero con un cambio que se definirá en el operador de vecindario.

También es importante recordar que la búsqueda local trabaja mejorando soluciones ya construidas, por lo que se utilizará como solución de partida la solución generada por la heurística ad-hoc descrita en el apartado anterior.

\subsection{Operador de vecindario}

El operador de vecindario que he decidido utilizar es un operador de vecino bastante simple para la representación del problema que estamos utilizando. Este operador escogerá un elemento $i$ de la solución tal que $x[i] = 1$, es decir, esté seleccionado, y otro elemento $j$ de la solución tal que $x[j] = 0$, y simplemente se dejará de seleccionar el elemento $i$ y se pasará a seleccionar $j$, y esa nueva solución será el vecino generado. El operador de vecino también tendrá en cuenta los vecinos ya generados, para no repetirlos


\begin{lstlisting}[language=python]
Operador_vecino(x, parejas_generadas):
	vecino <- x
	i <- numero entre 0 y la longitud de x tal que x[i] = 1
	j <- numero entre 0 y la longitud de x tal que x[j] = 0

	Mientras la pareja i,j esté en parejas_generadas:
		i <- numero entre 0 y la longitud de x tal que x[i] = 1
		j <- numero entre 0 y la longitud de x tal que x[j] = 0

	vecino[i] <- 0
	vecino[j] <- 1

	parejas_generadas <- insertar la pareja i,j y la pareja j, i

	Devolver la solucion vecino y parejas_generadas
\end{lstlisting}

\newpage

\subsection{Algoritmo de búsqueda local}


Una vez tenemos el operador de vecino, la búsqueda local tendría la siguiente forma:

\begin{lstlisting}[language=python]
Busqueda local(n, m, d):
	x <- Heuristica Ad-hoc(n, m, d)
	hay_mejora <- 1

	Mientras hay_mejora = 1:
		parejas_generadas <- conjunto vacio
		hay_mejora <- 0

		Mientras queden parejas sin generar en parejas_generadas y hay_mejora = 0:
			vecino, parejas_generadas <- Operador_vecino(x, parejas_generadas)
			Si MD(vecino) > MD(x):
				hay_mejora <- 1
				x <- vecino
	Devolver la solucion x
\end{lstlisting}

Como vemos, esta búsqueda parte de la solución inicial de la heurística ad-hoc, y simplemente cada vez que encuentra una nueva solución mejor, parte de esa nueva solución, comienza a generar vecinos hasta encontrar otra solución mejor y volver a comenzar de nuevo con esa nueva solución. En el caso de que genere todos los vecinos posibles y no encuentre una mejor solución, la búsqueda se detendría y devolvería la mejor solución encontrada hasta el momento.
